\documentclass[10pt,a4paper]{article}
\usepackage[utf8]{inputenc}
\usepackage[T1]{fontenc}
\usepackage{amssymb}
\usepackage{graphicx}
\usepackage{mathtools}
\usepackage{amsthm}
\usepackage{pgfplots}
\newtheorem{definition}{Definition}
\begin{document}
	$\frac{d}{dx}(\sin{x}) = \cos{x}$ \space $\frac{d}{dx}(\cos{x}) = -\sin{x}$
	\\	$\frac{d}{dx}(\tan{x}) = \sec^{2}{x}$	$\frac{d}{dx}(\csc{x}) = \csc{x}\cot{x}$
	\\	$\frac{d}{dx}(\sec{x})= \sec{x}\tan{x}$	$\frac{d}{dx}(\cot{x}) = -\csc^{2}{x}$
	\\	$\frac{d}{dx}(\sin^{-1}{x}) = \frac{1}{\sqrt{1-x^{2}}}$	$\frac{d}{dx}(\cos^{-1}{x}) = -\frac{1}{\sqrt{1-x^{2}}}$
	\\	$\frac{d}{dx}(\tan^{-1}{x}) = \frac{1}{1+x^{2}}$ $\frac{d}{dx}(\cot^{-1}{x}) = -\frac{1}{1+x^{2}}$
	\\ $\frac{d}{dx}(\sec^{-1}{x}) = \frac{1}{x\sqrt{x^{2}-1}}$ $\frac{d}{dx}(\csc^{-1}{x}) = -\frac{1}{x\sqrt{x^{2}-1}}$
	
Anti-Derivatives to Remember:
\\ $cf(x) \rightarrow cF(x)$  /  $f(x) + g(x) \rightarrow F(x) + G(x)$
\\ $x^{n}(n \neq -1) \rightarrow \frac{x^{n+1}}{n+1}$ /  $\frac{1}{x} \rightarrow \ln{|x|}$
\\ $e^{x} \rightarrow e^{x}$ /  $b^{x} \rightarrow \frac{b^{x}}{\ln{b}}$
\\ $\cos{x} \rightarrow \sin{x} $ /  $\sin{x} \rightarrow -\cos{x}$
\\ $\sec^{2}{x} \rightarrow \tan{x}$  / $\sec{x} \tan{x} \rightarrow \sec{x}$
\\ $\frac{1}{\sqrt{1-x^{2}}} \rightarrow \sin^{-1}{x}$ /  $\frac{1}{1+x^{2}} \rightarrow \tan^{-1}{x}$
\\ $\cosh{x} \rightarrow \sinh{x}$ /  $\sinh{x} \rightarrow \cosh{x}$	

 Indeterminate Forms:
\\ $\frac{0}{0}$ / $\frac{\infty}{\infty}$
\\ $\frac{-\infty}{-\infty}$ / $0^{0}$
\\ $ \infty^{0}$ / $1^{\infty}$
Use logs to convert to a workable form for L'Hospital's Rule
		\begin{definition}
		The area $A$ of the region $S$ that lies under the graph of the continuous function $f$ is the limit of the sum or the areas of approximating rectangles:
		\begin{center}
			$A=\lim_{n\rightarrow \infty}{[f(x-1)\Delta{x}+f(x_2)\Delta{x} + f(x_n)\Delta{x}]}$
		\end{center}
	\end{definition} 
		Sigma notation or a Riemann sum is read as $\sum_{i=1}^{n} f(x_{1})\Delta{x} = f(x_{1}\Delta{x} + f(x_{n}\Delta{x})$
	\\			$n$ is where the ending x-value of the last rectangle
	\\			$i=1$ is where the starting x-value that the rectangle will be
	\subsection{The Definite Integral}
	\begin{definition}
		If $f$ is a function defined for $a\geq x \geq b$, we divide the interval $[a,b]$ into $n$ sub-intervals of equal width $\Delta{x} = \frac{b-a}{n}$. Letting $x_{0}(=a),x_{1}, x_{2}, x_{n}(=b)$ be the endpoints for the sub-intervals, while also letting $x_{1}^{*} , x_{2}^{*} , x_{n}^{*}$ be sample points in each of the subintervals, which means that $x_{i}^{*}$ is in the subinterval $[x_{i-1}, x_{i}]$
		\\			This leads to the definite integral from $a$ to $b$ being $\int_{a}^{b} f(x)dx = \lim_{n \rightarrow \infty}{\sum_{i=1}^{n}{f(x_{i}^{*}\Delta{x})}}$
		\\			\textbf{THE LIMIT OF THIS FUNCTION MUST EXIST FOR THE ABOVE EQUATION TO WORK}
	\end{definition}
	
	\subsubsection{Evaluating Integrals}
	\begin{enumerate}
		\item $\sum_{i=1}^{n}{i} = \frac{n(n+1)}{2}$
		\item $\sum_{i=1}^{n}{i^{2}} = \frac{n(n+!)(2n+1)}{6}$
		\item $\sum_{i=1}^{n}{i^{3}} = [\frac{n(n+1)}{2}]^{2}$
	\end{enumerate}
	Rules for working with Sigma Notation
	\begin{enumerate}
		\item $\sum_{i=1}^{n}{c} = nc$
		\item $\sum_{i=1}^{n}{ca_{i}} = c\sum_{i=1}^{n}{a_{i}}$
		\item $\sum_{i=1}^{n}{(a_{i}+b_{i})} = \sum_{i=1}{n}{a_{i}} + \sum_{i=1}{n}{b_{i}}$
		\item $\sum_{i=1}^{n}{(a_{i}-b_{i})} = \sum_{i=1}^{n}{a_{i}} - \sum_{i=1}^{n}{b_{i}}$
	\end{enumerate}
			Since the limit of a Riemann sum works no matter if $a<b$ or $a>b$ we can assume that $\int_{b}^{a}{f(x)dx} = -\int_{b}^{a}{f(x)dx}$
	\\			Similar to the derivative rule, if $a=b$ then the change in x(or $\Delta{x})=0$ so the integral evaluates to zero
	\\			Here are some of the basic properties of the integral:
	\begin{enumerate}
		\item $\int_{b}^{a}{c dx} = c(b-a)$ when $c$ is a constant
		\item $\int_{b}^{a}{f(x)+g(x)}dx = \int_{b}^{a}{f(x)dx} + \int_{b}^{a}{g(x)dx}$
		\item $\int_{b}^{a}{cf(x)dx} = c\int_{b}^{a}{f(x)dx}$ when $c$ is a constant
		\item $\int_{b}^{a}{[f(x)-g(x)]dx} = \int_{b}^{a}{f(x)dx} - \int_{b}^{a}{g(x)dx}$
	\end{enumerate}
	\subsubsection{The Fundamental Theorem of Calculus}
	Part 1: If $f$ is continuous on $[a,b]$ then the function $g$ defined by
	\begin{center}
		$g(x) = \int_{a}^{x}{f(t)dt}$ when $a \leq x \leq b$
	\end{center}
	As long as the function is continuous along its interval and differentiable along that same interval
	\\		Part 2: If $f$ is continuous on $[a,b]$ then: 
	\begin{center}
		$\int_{a}^{b}{f(x)dx} = F(b) - F(a)$
	\end{center}
	Meaning that as long as $F$ is any anti derivative of $f$, $F'(x) = f(x)$

\end{document}