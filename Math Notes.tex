\documentclass[10pt,a4paper]{article}
\usepackage[utf8]{inputenc}
\usepackage[T1]{fontenc}
\usepackage{amssymb}
\usepackage{graphicx}
\usepackage{mathtools}
\usepackage{amsthm}
\usepackage{nameref}
\usepackage{thmtools}
\newtheorem{theorem}{Theorem}
\newtheorem{definition}{Definition}
\title{Math Notes}
\begin{document}
	\section{Chapter 1}
	
	\section{Chapter 2}
	
	\section{Chapter 3}
		\subsection{Derivatives of Polynomials and Exponential Functions}
		\subsubsection{Power Rule}
		
		General Power Rule: $\frac{d}{dx}(x^{n}) = nx^{n-1}$
		
		\subsubsection{Constant Multiple Rule}
		If c is a constant and f is a differentiable function: 
		\begin{center}
			$\frac{d}{dx}cf(x) = c\frac{d}{dx}f(x)$
		\end{center}
		\subsubsection{Sum Rule}
		If f and g are both differentiable:
		\begin{center}
			$\frac{d}{dx} [f(x)+ g(x)] = \frac{d}{dx}f(x) + \frac{d}{dx}g(x)$
		\end{center}
		\subsubsection{Difference Rule}
		If f and g are both differentiable: 
		\begin{center}
			$\frac{d}{dx}[f(x) - g(x)] = \frac{d}{dx}f(x) - \frac{d}{dx}g(x)$
		\end{center}
		\subsubsection{Product Rule}
		If f and g are both differentiable: 
		\begin{center}
			$\frac{d}{dx}[f(x)g(x)] = f(x)\frac{d}{dx}[g(x)] + g(x)\frac{d}{dx}[f(x)]$
		\end{center}
		\subsubsection{Quotient Rule}
		If f and g are differentiable:
		\begin{center}
			$\frac{d}{dx}[\frac{f(x)}{g(x)}] = \frac{g(x)\frac{d}{dx}[f(x)]-f(x)\frac{d}{dx}[g(x)]}{[g(x)]^{2}}$
		\end{center}
		\subsection{The Product and Quotient Rules}
		\subsection{Derivatives of Trigonometric Functions}
			$\frac{d}{dx}(\sin{x}) = \cos{x}$
		\\	$\frac{d}{dx}(\cos{x}) = -\sin{x}$
		\\	$\frac{d}{dx}(\tan{x}) = \sec^{2}{x}$
		\\	$\frac{d}{dx}(\csc{x}) = \csc{x}\cot{x}$
		\\	$\frac{d}{dx}(\sec{x})= \sec{x}\tan{x}$
		\\	$\frac{d}{dx}(\cot{x}) = -\csc^{2}{x}$
		\\	$\frac{d}{dx}(\sin^{-1}{x}) = \frac{1}{\sqrt{1-x^{2}}}$
		\\	$\frac{d}{dx}(\cos^{-1}{x}) = -\frac{1}{\sqrt{1-x^{2}}}$
		\\	$\frac{d}{dx}(\tan^{-1}{x}) = \frac{1}{1+x^{2}}$
		\\ $\frac{d}{dx}(\cot^{-1}{x}) = -\frac{1}{1+x^{2}}$
		\\ $\frac{d}{dx}(\sec^{-1}{x}) = \frac{1}{x\sqrt{x^{2}-1}}$
		\\ $\frac{d}{dx}(\csc^{-1}{x}) = -\frac{1}{x\sqrt{x^{2}-1}}$

	
	\subsection{Chain Rule}
		Given the function $\frac{d}{dx}f(g(x))$, $ \frac{d}{dx} = f'(g(x))*g'(x)$
\\		If \textit{g} is differentiable at \textit{x} and \textit{f} is differentiable at $g(x)$ then the composite function $F = f*g$ defined by $F(x) = f(g(x))$ is differentiable at $x$ and that $F'$is given by the product:
		\begin{center}
			$F'(x) = f'(g(X)) * g'(x)$	
		\end{center}
		In Leibniz notation, if $y=f(u)$ and $u=g(x)$ are both differentiable functions then:
		\begin{center}
			$\frac{dy}{dx} = \frac{dy}{du}\frac{du}{dx}$
		\end{center}
		\begin{theorem}
			\textbf{Power Rule Combined With Chain Rule} If \textit{n} is any real number and $u=g(x)$ is differentiable then,
			\begin{center}
				$\frac{d}{dx}(u^{n}) = nu^{n-1}\frac{du}{dx}$
			\end{center}
			As such, $\frac{d}{dx}[g(x)]^{n} = n[g(x)]^{n-1} * g'(x)$
		\end{theorem}
	\subsection{Implicit Differentiation}
	\begin{definition}
		\textbf{Implicit Differentiation}:
		\\ The process of differentiating both sides of the equation with respect to x and then solving the resulting equation for $y'$
	\end{definition}
	Essentially, if the function says to take the derivative with respect to x, and there is a variable other than x present in the equation, derive the equation using normal differentiation rules, and at a $\frac{dn}{dx}$ where n is the variable letter present in the equation
	\subsection{Derivatives of Logarithmic Functions}
		General Form: $\frac{d}{dx}(log_{b}{x}) = \frac{1}{x\ln{b}}$
		Proof: Let $y=\log_{b}{x}$ Then
		\begin{center}
			$b^{y}=x$
		\end{center}
		Using the general form for taking the derivative, you get:
		\begin{center}
			$b^{y}(\ln{b})\frac{dy}{dx}=1$
			\\so
			\\$\frac{dy}{dx}=\frac{1}{b^{y}\ln{b}} = \frac{1}{x\ln{b}}$
		\end{center}
		Taking the Derivative of $y=\ln{x}$:
		\\Using the above General Form, you can solve for the derivative of $y=\ln{x}$:
		\begin{center}
			$\frac{d}{dx}(\ln{x})=\frac{1}{x}$
		\end{center}
		\textbf{IMPORTANT}:
		\\When taking the derivative of a logarithmic function, such as the derivative of $y=\ln{x^{3}+1}$ you will have to use chain rule.
		\\$\frac{d}{dx}(\ln{u}) = \frac{1}{u}\frac{du}{dx}$ and $\frac{d}{dx}[\ln{g(x)}] = \frac{g'(x)}{g(x)}$
		\\Another Derivative to remember is the derivative of $\ln|x|$ which derives to $\frac{1}{x}$
		\subsubsection{Using Logarithmic Differentiation}
		1. Take natural logarithms of both sides of an equation $y=f(x)$ and use the Laws of Logarithms to simplify
		\\2. Differentiate implicitly with respect to x.
		\\3. Splve the resulting equation for $y'$
		For example, let $y=x^{n}$
		\begin{center}
			$\ln{|y|} = \ln{|x|^{n}}=n\ln{x}$ when $x \neq 0$
		\end{center}
		For Logarithmic differentiation to be used, both the \textbf{base} and the \textbf{exponent} must not be constants
		\begin{definition}
			$e=\lim_{x\rightarrow 0}{(1+x)^{\frac{1}{x}}}$ 
			\\which when evaluated, defines e as $e\approx 2.7182818$
			\\Another form to define e is: 
			$e=\lim_{n\rightarrow\infty}{(1+\frac{1}{n})^{n}}$
		\end{definition}
	\subsection{Rates of Change in the Natural and Social Sciences}
	Rates of change: 
	\\If x changes from $\Delta{x} = x_{2}-x_{1}$
	\\Which means the change in y is $\Delta{y} = f(x_{2})-f(x_{1})$
	\\The difference quotient, which will give us the average rate of change in y with respect to x is defined as
	\begin{center} 
		$\frac{\Delta{y}}{\Delta{x}}= \frac{f(x_{2})-f(x_{1})}{x_{2}-x_{1}}$
	\end{center}
	Finding the instantaneous rate of change of y with respect to x is defined as:
	\begin{center}
		$\frac{dy}{dx}= \lim_{\Delta{x}\rightarrow 0}{\frac{\Delta{y}}{\Delta{x}}}$
	\end{center}
	\subsection{Exponential Growth and Decay}
	If $y(t)$ is the value of a quantity $y$ at time $t$ and if the rate of change of $y$ with respect to $t$ is proportional to its size $y(t)$ at any time, then:
	\begin{center}
		$\frac{dy}{dt} = ky$
	\end{center}	
	$y'(t)=C(ke^{kt})=k(Ce^{kt})=ky(t)$
	where k is the rate at which the function is growing or decaying
	\\C is a constant
	\\T is time
\pagebreak
	\subsection{Related Rates}
		General Process for solving related rates problems:
\\		1. Read the problem carefully
\\		2. Draw a diagram if possible
\\		3. Introduce notation. Assign symbols to all quantities that are functions of time
\\		4. Express the given information and the required rates in terms of derivatives
\\		5. Write an equation that relates the various quantities of the problems. If necessary, use the geometry of the situation to eliminate one of the variable by substitution
\\		6. Use the Chain Rule to differentiate both sides of the equation with respect to t
\\		7. Substitute the given information into the resulting equation and solve for the unknown rate.


\section{Chapter 4}
\subsection{Minimum and Maximum Values}
\begin{definition}
	Let $c$ be a number in the domain $D$ of a function $f$. Then $f(c)$ is the:
	\begin{center}
		\textbf{Absolute maximum} value of $f$ on $D$ if $f(c) \geq f(x) $ for all $x$ in $D$.
\\		\textbf{Absolute minimum} value of $f$ on $D$ if $f(c) \leq f(x) $ for all $x$ in $D$.
	\end{center}
\end{definition}
\begin{definition}
	The number $f(c)$ is a
	\begin{center}
		\textbf{Local Maximum} value of $f$ if $f(c) \geq f(x)$ when $x$ is near $c$.
\\		\textbf{Local Minimum} values of $f$ if $f(c) \leq f(x)$ when $x$ is near $c$.
	\end{center}
\end{definition}
\begin{theorem}
	\textbf{Extreme Value Theorem}
\\	If $f$ is continuous on a closed interval $[a,b]$ then $f$ attains an absolute maximum values $f(c)$ and an absolute minimum value $f(d)$ at some numbers $c$ and $d$ in $[a,b]$.
\end{theorem}
\begin{theorem}
	\textbf{Fermat's Theorem}
\\	If $f$ has a local maximum or minimum at $c$, and if $f'(c)$ exists, then $f'(c)=o$.
\\If $f$ has a local maximum or minimum at $c$, then $c$ is a critical number of $f$.
\end{theorem}
\begin{definition}
	A \textbf{critical number} of a function $f$ is a number $c$ c in the domain of $f$ such that either $f'(c)=0$ or $f'(c)$ does not exist.
\end{definition}
\subsubsection{Finding the Absolute Max and Min Values using the Closed Interval Method}
To find the \textit{absolute} maximum and minimum values of a continuous function $f$ on a closed interval $[a,b]$:
\\1. Find the values of $f$ at the critical numbers of $f$ in $(a,b)$.
\\2. Find the values of $f$ at the endpoints of the interval.
\\3. The largest of the values from Step 1 and 2 is the absolute maximum value; the smallest of these values is the absolute minimum value.

\subsection{Mean Value Theorem}
\begin{theorem}
	\textbf{Role's Theorem}:
	\\Let \textit{f} be a function that satisfies the following three hypotheses:
	\\1. \textit{f} is continuous on the closed interval $[a,b]$
	\\2. \textit{f} is differentiable on the open interval $(a,b)$
	\\3. $f(a) = f(b)$
	\\ Then there is a number \textit{c} in $(a,b)$ such that $f'(c)=0$	
\end{theorem}
\begin{theorem}
	\textbf{Mean Value Theorem}:
	\\Let $f$ be a function that satisfies the following hypotheses:
	\\1. $f$ is continuous on the closed interval $[a,b]$
	\\2. $f$ is differentiable on the open interval $(a,b)$
	\\ Then there is a number $c$ in $(a,b)$ such that:
	\begin{center}
		$f'(c)= \frac{f(b)-f(a)}{b-a}$
		\\or equivalently
		\\$f(b)-f(a) = f'(c)(b-a)$
	\end{center}
\end{theorem}
The slope of the secant line $AB$ is $m_{ab}=\frac{f(b)-f(a)}{b-a}$
\end{document}