\documentclass[10pt,a4paper]{article}
\usepackage[utf8]{inputenc}
\usepackage[T1]{fontenc}
\usepackage{amssymb}
\usepackage{graphicx}
\usepackage{mathtools}
\usepackage{amsthm}
\usepackage{nameref}
\usepackage{thmtools}
\usepackage{tikz}
\usepackage{pgfplots}
\pgfplotsset{compat=1.18}
\usepgfplotslibrary{units}
\newtheorem{theorem}{Theorem}
\newtheorem{definition}{Definition}
\title{Math Notes}
\begin{document}
	\section{Chapter 1}
	
	\section{Chapter 2}
	
	\section{Chapter 3}
		\subsection{Derivatives of Polynomials and Exponential Functions}
		\subsubsection{Power Rule}
		
		General Power Rule: $\frac{d}{dx}(x^{n}) = nx^{n-1}$
		
		\subsubsection{Constant Multiple Rule}
		If c is a constant and f is a differentiable function: 
		\begin{center}
			$\frac{d}{dx}cf(x) = c\frac{d}{dx}f(x)$
		\end{center}
		\subsubsection{Sum Rule}
		If f and g are both differentiable:
		\begin{center}
			$\frac{d}{dx} [f(x)+ g(x)] = \frac{d}{dx}f(x) + \frac{d}{dx}g(x)$
		\end{center}
		\subsubsection{Difference Rule}
		If f and g are both differentiable: 
		\begin{center}
			$\frac{d}{dx}[f(x) - g(x)] = \frac{d}{dx}f(x) - \frac{d}{dx}g(x)$
		\end{center}
		\subsubsection{Product Rule}
		If f and g are both differentiable: 
		\begin{center}
			$\frac{d}{dx}[f(x)g(x)] = f(x)\frac{d}{dx}[g(x)] + g(x)\frac{d}{dx}[f(x)]$
		\end{center}
		\subsubsection{Quotient Rule}
		If f and g are differentiable:
		\begin{center}
			$\frac{d}{dx}[\frac{f(x)}{g(x)}] = \frac{g(x)\frac{d}{dx}[f(x)]-f(x)\frac{d}{dx}[g(x)]}{[g(x)]^{2}}$
		\end{center}
		\subsection{The Product and Quotient Rules}
		\subsection{Derivatives of Trigonometric Functions}
			$\frac{d}{dx}(\sin{x}) = \cos{x}$
		\\	$\frac{d}{dx}(\cos{x}) = -\sin{x}$
		\\	$\frac{d}{dx}(\tan{x}) = \sec^{2}{x}$
		\\	$\frac{d}{dx}(\csc{x}) = \csc{x}\cot{x}$
		\\	$\frac{d}{dx}(\sec{x})= \sec{x}\tan{x}$
		\\	$\frac{d}{dx}(\cot{x}) = -\csc^{2}{x}$
		\\	$\frac{d}{dx}(\sin^{-1}{x}) = \frac{1}{\sqrt{1-x^{2}}}$
		\\	$\frac{d}{dx}(\cos^{-1}{x}) = -\frac{1}{\sqrt{1-x^{2}}}$
		\\	$\frac{d}{dx}(\tan^{-1}{x}) = \frac{1}{1+x^{2}}$
		\\ $\frac{d}{dx}(\cot^{-1}{x}) = -\frac{1}{1+x^{2}}$
		\\ $\frac{d}{dx}(\sec^{-1}{x}) = \frac{1}{x\sqrt{x^{2}-1}}$
		\\ $\frac{d}{dx}(\csc^{-1}{x}) = -\frac{1}{x\sqrt{x^{2}-1}}$

	
	\subsection{Chain Rule}
		Given the function $\frac{d}{dx}f(g(x))$, $ \frac{d}{dx} = f'(g(x))*g'(x)$
\\		If \textit{g} is differentiable at \textit{x} and \textit{f} is differentiable at $g(x)$ then the composite function $F = f*g$ defined by $F(x) = f(g(x))$ is differentiable at $x$ and that $F'$is given by the product:
		\begin{center}
			$F'(x) = f'(g(X)) * g'(x)$	
		\end{center}
		In Leibniz notation, if $y=f(u)$ and $u=g(x)$ are both differentiable functions then:
		\begin{center}
			$\frac{dy}{dx} = \frac{dy}{du}\frac{du}{dx}$
		\end{center}
		\begin{theorem}
			\textbf{Power Rule Combined With Chain Rule} If \textit{n} is any real number and $u=g(x)$ is differentiable then,
			\begin{center}
				$\frac{d}{dx}(u^{n}) = nu^{n-1}\frac{du}{dx}$
			\end{center}
			As such, $\frac{d}{dx}[g(x)]^{n} = n[g(x)]^{n-1} * g'(x)$
		\end{theorem}
	\subsection{Implicit Differentiation}
	\begin{definition}
		\textbf{Implicit Differentiation}:
		\\ The process of differentiating both sides of the equation with respect to x and then solving the resulting equation for $y'$
	\end{definition}
	Essentially, if the function says to take the derivative with respect to x, and there is a variable other than x present in the equation, derive the equation using normal differentiation rules, and at a $\frac{dn}{dx}$ where n is the variable letter present in the equation
	\subsection{Derivatives of Logarithmic Functions}
		General Form: $\frac{d}{dx}(log_{b}{x}) = \frac{1}{x\ln{b}}$
		Proof: Let $y=\log_{b}{x}$ Then
		\begin{center}
			$b^{y}=x$
		\end{center}
		Using the general form for taking the derivative, you get:
		\begin{center}
			$b^{y}(\ln{b})\frac{dy}{dx}=1$
			\\so
			\\$\frac{dy}{dx}=\frac{1}{b^{y}\ln{b}} = \frac{1}{x\ln{b}}$
		\end{center}
		Taking the Derivative of $y=\ln{x}$:
		\\Using the above General Form, you can solve for the derivative of $y=\ln{x}$:
		\begin{center}
			$\frac{d}{dx}(\ln{x})=\frac{1}{x}$
		\end{center}
		\textbf{IMPORTANT}:
		\\When taking the derivative of a logarithmic function, such as the derivative of $y=\ln{x^{3}+1}$ you will have to use chain rule.
		\\$\frac{d}{dx}(\ln{u}) = \frac{1}{u}\frac{du}{dx}$ and $\frac{d}{dx}[\ln{g(x)}] = \frac{g'(x)}{g(x)}$
		\\Another Derivative to remember is the derivative of $\ln|x|$ which derives to $\frac{1}{x}$
		\subsubsection{Using Logarithmic Differentiation}
		1. Take natural logarithms of both sides of an equation $y=f(x)$ and use the Laws of Logarithms to simplify
		\\2. Differentiate implicitly with respect to x.
		\\3. Splve the resulting equation for $y'$
		For example, let $y=x^{n}$
		\begin{center}
			$\ln{|y|} = \ln{|x|^{n}}=n\ln{x}$ when $x \neq 0$
		\end{center}
		For Logarithmic differentiation to be used, both the \textbf{base} and the \textbf{exponent} must not be constants
		\begin{definition}
			$e=\lim_{x\rightarrow 0}{(1+x)^{\frac{1}{x}}}$ 
			\\which when evaluated, defines e as $e\approx 2.7182818$
			\\Another form to define e is: 
			$e=\lim_{n\rightarrow\infty}{(1+\frac{1}{n})^{n}}$
		\end{definition}
	\subsection{Rates of Change in the Natural and Social Sciences}
	Rates of change: 
	\\If x changes from $\Delta{x} = x_{2}-x_{1}$
	\\Which means the change in y is $\Delta{y} = f(x_{2})-f(x_{1})$
	\\The difference quotient, which will give us the average rate of change in y with respect to x is defined as
	\begin{center} 
		$\frac{\Delta{y}}{\Delta{x}}= \frac{f(x_{2})-f(x_{1})}{x_{2}-x_{1}}$
	\end{center}
	Finding the instantaneous rate of change of y with respect to x is defined as:
	\begin{center}
		$\frac{dy}{dx}= \lim_{\Delta{x}\rightarrow 0}{\frac{\Delta{y}}{\Delta{x}}}$
	\end{center}
	\subsection{Exponential Growth and Decay}
	If $y(t)$ is the value of a quantity $y$ at time $t$ and if the rate of change of $y$ with respect to $t$ is proportional to its size $y(t)$ at any time, then:
	\begin{center}
		$\frac{dy}{dt} = ky$
	\end{center}	
	$y'(t)=C(ke^{kt})=k(Ce^{kt})=ky(t)$
	where k is the rate at which the function is growing or decaying
	\\C is a constant
	\\T is time
\pagebreak
	\subsection{Related Rates}
		General Process for solving related rates problems:
\\		1. Read the problem carefully
\\		2. Draw a diagram if possible
\\		3. Introduce notation. Assign symbols to all quantities that are functions of time
\\		4. Express the given information and the required rates in terms of derivatives
\\		5. Write an equation that relates the various quantities of the problems. If necessary, use the geometry of the situation to eliminate one of the variable by substitution
\\		6. Use the Chain Rule to differentiate both sides of the equation with respect to t
\\		7. Substitute the given information into the resulting equation and solve for the unknown rate.


\section{Chapter 4}
\subsection{Minimum and Maximum Values}
\begin{definition}
	Let $c$ be a number in the domain $D$ of a function $f$. Then $f(c)$ is the:
	\begin{center}
		\textbf{Absolute maximum} value of $f$ on $D$ if $f(c) \geq f(x) $ for all $x$ in $D$.
\\		\textbf{Absolute minimum} value of $f$ on $D$ if $f(c) \leq f(x) $ for all $x$ in $D$.
	\end{center}
\end{definition}
\begin{definition}
	The number $f(c)$ is a
	\begin{center}
		\textbf{Local Maximum} value of $f$ if $f(c) \geq f(x)$ when $x$ is near $c$.
\\		\textbf{Local Minimum} values of $f$ if $f(c) \leq f(x)$ when $x$ is near $c$.
	\end{center}
\end{definition}
\begin{theorem}
	\textbf{Extreme Value Theorem}
\\	If $f$ is continuous on a closed interval $[a,b]$ then $f$ attains an absolute maximum values $f(c)$ and an absolute minimum value $f(d)$ at some numbers $c$ and $d$ in $[a,b]$.
\end{theorem}
\begin{theorem}
	\textbf{Fermat's Theorem}
\\	If $f$ has a local maximum or minimum at $c$, and if $f'(c)$ exists, then $f'(c)=o$.
\\If $f$ has a local maximum or minimum at $c$, then $c$ is a critical number of $f$.
\end{theorem}
\begin{definition}
	A \textbf{critical number} of a function $f$ is a number $c$ c in the domain of $f$ such that either $f'(c)=0$ or $f'(c)$ does not exist.
\end{definition}
\subsubsection{Finding the Absolute Max and Min Values using the Closed Interval Method}
To find the \textit{absolute} maximum and minimum values of a continuous function $f$ on a closed interval $[a,b]$:
\\1. Find the values of $f$ at the critical numbers of $f$ in $(a,b)$.
\\2. Find the values of $f$ at the endpoints of the interval.
\\3. The largest of the values from Step 1 and 2 is the absolute maximum value; the smallest of these values is the absolute minimum value.

\subsection{Mean Value Theorem}
\begin{theorem}
	\textbf{Role's Theorem}:
	\\Let \textit{f} be a function that satisfies the following three hypotheses:
	\\1. \textit{f} is continuous on the closed interval $[a,b]$
	\\2. \textit{f} is differentiable on the open interval $(a,b)$
	\\3. $f(a) = f(b)$
	\\ Then there is a number \textit{c} in $(a,b)$ such that $f'(c)=0$	
\end{theorem}
\begin{theorem}
	\textbf{Mean Value Theorem}:
	\\Let $f$ be a function that satisfies the following hypotheses:
	\\1. $f$ is continuous on the closed interval $[a,b]$
	\\2. $f$ is differentiable on the open interval $(a,b)$
	\\ Then there is a number $c$ in $(a,b)$ such that:
	\begin{center}
		$f'(c)= \frac{f(b)-f(a)}{b-a}$
		\\or equivalently
		\\$f(b)-f(a) = f'(c)(b-a)$
	\end{center}
\end{theorem}
The slope of the secant line $AB$ is $m_{ab}=\frac{f(b)-f(a)}{b-a}$

\subsection{How Derivatives Affect the Shape of the Graph}
\subsubsection{Increasing Decreasing Test}
a) If $f'(x) > 0$ on an interval, then $f$ is increasing on that interval
\\b) If $f'(x) < 0 $ on an interval, then $f$ is decreasing on that interval

\subsubsection{The First Derivative Test}
Suppose that $c$ is a critical number of a continuous function $f$.
\\
\\ a) If $f'$ changes from positive to negative at $c$, then $f$ has a local maximum at $c$.
\\ b) If $f'$ changes from negative to positive at $c$, then $f$ has a local minimum at $c$.
\\ c) If $f'$ is positive to the left and right of $c$, or negative to the left and right of $c$, then $f$ has no local maximum or minimum at $c$.

\begin{definition}
	If the graph of $f$ lies above all of its tangents on an interval $I$, then it is called \textbf{concave upward} on $I$. If the graph of $f$ lies below all of its tangents on $I$, it is called \textbf{concave downwards} on $I$.
\end{definition}
\subsubsection{Concavity Test}
(a) If $f"(x) > 0$ for all $x$ in $I$, then the graph of $f$ is concave upward on $I$.
(b) If $f"(x) < 0$ for all $x$ in $I$, then the graph of $f$ is concave downward on $I$.

\begin{definition}
	\textbf{Inflection Point}: A point $P$ on a curve $y=f(x)$ is called an inflection point if $f$ is continuous there and the curve changes from concave upward to concave downward or the concave downward to concave upward at $P$.
\end{definition}
\subsubsection{The Second Derivative Test}
Suppose $f"$ is continuous near $c$. 
\\(a) If $f'(c) = 0$ and $f"(c) >0$, then $f$ has a local minimum at $c$.
\\(b) If $f'(c) = 0$ and $f"(c) < 0$, then $f$ has a local maximum at $c$.

\subsection{Indeterminate Forms and L'Hospital's Rule}
\subsubsection{L'Hospital's Rule}
\textbf{IMPORTANT} You can only use L'Hospital's Rule when the limit is in an indeterminate form like $\frac{0}{0}$, $\frac{-\infty}{-\infty}$, or $\frac{\infty}{\infty}$
\\Suppose $f$ and $g$ are differentiable and $g'(x) \neq 0 $ on an open interval $I$ that contains $a$(except possibly at $a$). Suppose that:
\begin{center}
	$\lim_{x\rightarrow a}{f(x)} = 0$ and
	$\lim_{x\rightarrow a}{g(x)} = 0$
\\	or that
\\	$\lim_{x\rightarrow a}{f(x)} = \pm \infty $ and $\lim_{x \rightarrow a}{g(x)} = \pm \infty $
\end{center}
(In other words, we have an indeterminate form of type $\frac{0}{0}$ or $\frac{\infty}{\infty}$. Then: 
\begin{center}
	$\lim_{x\rightarrow a}{\frac{f(x)}{g(x)}} = \lim_{x \rightarrow a}{\frac{f'(x)}{g'(x)}}$
\end{center}
if the limit on the right side exists (or is $\infty $ or $-\infty$).
\subsubsection{Indeterminate Forms}
Indeterminate forms is when an equations evaluates into $\frac{0}{0}$, $\frac{\infty}{\infty}$, $\frac{-\infty}{-\infty}$
\\This even works for functions in a form like $f(x) - g(x)$
\\  \textbf{Indeterminate Powers}:
There are several indeterminate forms arise from the limit:
\begin{center}
	$\lim_{x\rightarrow a}{[f(x)]^{g(x)}}$
\end{center}
1.$\lim_{x \rightarrow a}{f(x)} = 0$ and $ \lim_{x \rightarrow a}{g(x)} = 0$ which evaluates to the type $0^{0}$
\\2. $\lim_{x \rightarrow a}{f(x)} = \infty$ and $\lim_{x \rightarrow a}{g(x)} = 0$ which evaluates to the type $\infty^{0}$
\\ 3. $\lim_{x \rightarrow a}{f(X)} = 1$ and $\lim_{x \rightarrow a}{g(x)} = \pm \infty $ which evaluates to the type $1^{\infty}$

Evaluate each of these cases by taking the natural logarithm: 
\\ So: $y=[f(x)]^{g(x)}$, then $\ln y = g(x) \ln f(x)$
\\ Or by converting to an exponential: $[f(x)]^{g(x)} = e^{g(x)\ln f(x)}$

\subsection{Summary of Curve Sketching}
	\subsubsection{Guidelines for Sketching a Curve}
		\begin{enumerate}
			\item Domain: Start by determining the domain $D$ of $f$
			\item Intercepts: Find the x and y intercepts if possible
			\item Determine Symmetry
				\begin{enumerate}
					\item Even Function:If $f(-x) = f(x)$ for all $x$ in $D$, that means the graph is symmetric and you can graph the function by graphing half than reflecting across the y-axis
					\item Odd Function: If $f(-x) = -f(x)$ for all $x$ in $D$, the curve is symmetric around the origin. You can graph the function more easily by taking one half of the function and rotating that line 180$\deg$ around the origin
					\item Periodic Function: If $f(x+p) = f(x)$ for all $x$ in $D$, when $p$ is a positive constant, and the smallest such number $p$ is the period. Example is the function $y=sin{x}$
				\end{enumerate}
			\item \textbf{Asymptotes}
				\begin{enumerate}
					\item \textbf{Horizontal Asymptotes} \quad If either $\lim_{x \rightarrow -\infty}{f(x)}=L$ or \linebreak $\lim_{x \rightarrow \infty}{f(x)} = L$ then it is a horizontal asymptote
					
					\item \textbf{Vertical Asymptotes} \quad  The line $x=a$ is a vertical asymptote if:
						\begin{center}
							$\lim_{x \rightarrow a^{+}}{f(x)} = \infty$\quad or \quad $\lim_{x \rightarrow a^{-}}{f(x)}=\infty $
\\							$\lim_{x \rightarrow a^{+}{f(x)}} = -\infty $\quad or \quad $\lim_{x \rightarrow a^{-}}{f(x)}= - \infty $
						\end{center}
				\end{enumerate}

			\item \textbf{Intervals of Increase or Decrease} Use the Increasing/Decreasing Test. Compute $f'(x)$ and find the intervals on which $f'(x)$ is positive($f$ is increasing) and the intervals on which $f'(x)$ is negative($f$ is decreasing)
			
			\item \textbf{Local Maximum and Minimum Values} Find the critical numbers of $f$[the numbers $c$ where $f'(c)=0$ or $f'(c)$ does not exist]. Then use First Derivative Test. If $f'$ changes from positive to negative at $c$, then $f(c)$ is a local maximum. If $f'$ changes from negative to positive at $c$, then $f(c)$ is a local minimum.
			
			\item \textbf{Concavity and Points of Inflection} Compute $f"(x)$ and use the Concavity Test. The curve is concave upward where $f"(x) > 0$ and concave downward where $f"(x) < 0$. Inflection points occur where the direction of concavity changes.
			
			\item \textbf{Sketch the Curve} Take all of the info gathered from the previous steps and combine it into one cohesive graph.
	\end{enumerate}	
	\subsubsection{Slant Asymptotes}
		Some curves have asymptotes that are oblique, i.e neither horizontal nor vertical.
		\\ If $\lim_{x \rightarrow \infty}{f(x)-(mx+b)} = 0$ where $m \neq 0$ then the line $mx+b$ is a slant asymptote.
\subsection{Graphing With Calculus and a Calculator}
Not Applicable due to Professor Wong not letting us use graphing calculators.

\subsection{Optimization Problems}
	\subsubsection{Steps in Solving Optimization Problems}
		\begin{enumerate}
			\item \textbf{Understand the Problem} Ask these kinds of questions: What is the unknown? What are the given quantities? What are the given conditions?
			
			\item \textbf{Draw a Diagram} In most problems it is useful to draw diagram and identify the given and required quantities on the diagram
			
			\item \textbf{Introduce Notation} Assign a symbol to the quantity that is to be maximized or minimized. Also assign symbols to other quantities that are unknown and use them in the diagram.
			
			\item Express the Symbol(I will be using K) used for the unknown quantity in terms of some of the other symbols chosen in the previous step
			
			\item If $K$ was expressed as a function of more than one variable, use the given information to find relationships between the variables. What you are trying to do is solve for a single variable, so find relations that allow you to simplify the equations to solve for the given variable, in this case $K$
			
			\item Find the absolute maximum or minimum values of $f$ using the Extreme Value theorem(Section 4.1 Theorem 2)
		\end{enumerate}
			
\subsection{Newton's Method}
	Ping me on discord if I'm wrong, but I don't think we covered this section in class.
	
\subsection{AntiDerivatives}
	\begin{definition}
		\textbf{Antiderivative}: The anti-derivative of a function is where in an interval notated as $I$, $F'(x) = f(x)$ for all $x$ in $I$.
	\end{definition}
	When you are taking the anti-derivative of a function, remember to include a $+C$ at the end of it, as when you take the anti-derivative, a constant is included in the new equation. The constant can be any number, which is why $C$ stands in for any number, as $C$ could be any value, yet the function will always be true.
	\begin{theorem}
		General Form of the anti-derivative of the function $F'(x)$ is defined as $F(x) + C$ where $C$ is a constant
	\end{theorem}
	
	\subsubsection{Key Anti-Derivatives to Remember}
		\begin{enumerate}
				\item $cf(x) \rightarrow cF(x)$
				\item $f(x) + g(x) \rightarrow F(x) + G(x)$
				\item $x^{n}(n \neq -1) \rightarrow \frac{x^{n+1}}{n+1}$
				\item $\frac{1}{x} \rightarrow \ln{|x|}$
				\item $e^{x} \rightarrow e^{x}$
				\item $b^{x} \rightarrow \frac{b^{x}}{\ln{b}}$
				\item $\cos{x} \rightarrow \sin{x} $
				\item $\sin{x} \rightarrow -\cos{x}$
				\item $\sec^{2}{x} \rightarrow \tan{x}$
				\item $\sec{x} \tan{x} \rightarrow \sec{x}$
				\item $\frac{1}{\sqrt{1-x^{2}}} \rightarrow \sin^{-1}{x}$
				\item $\frac{1}{1+x^{2}} \rightarrow \tan^{-1}{x}$
				\item $\cosh{x} \rightarrow \sinh{x}$
				\item $\sinh{x} \rightarrow \cosh{x}$
		\end{enumerate}


 


	
\end{document}
