\documentclass[10pt,a4paper]{article}
\usepackage[utf8]{inputenc}
\usepackage[T1]{fontenc}
\usepackage{amssymb}
\usepackage{graphicx}
\usepackage{mathtools}
\usepackage{amsthm}
\usepackage{nameref}
\usepackage{thmtools}
\newtheorem{theorem}{Theorem}
\newtheorem{definition}{Definition}
\title{Math Notes}
\begin{document}
	\section{Chapter 1}
	
	\section{Chapter 2}
	
	\section{Chapter 3}
		\subsection{Derivatives of Polynomials and Exponential Functions}
		\subsubsection{Power Rule}
		
		General Power Rule: $\frac{d}{dx}(x^{n}) = nx^{n-1}$
		
		\subsubsection{Constant Multiple Rule}
		If c is a constant and f is a differentiable function: 
		\begin{center}
			$\frac{d}{dx}cf(x) = c\frac{d}{dx}f(x)$
		\end{center}
		\subsubsection{Sum Rule}
		If f and g are both differentiable:
		\begin{center}
			$\frac{d}{dx} [f(x)+ g(x)] = \frac{d}{dx}f(x) + \frac{d}{dx}g(x)$
		\end{center}
		\subsubsection{Difference Rule}
		If f and g are both differentiable: 
		\begin{center}
			$\frac{d}{dx}[f(x) - g(x)] = \frac{d}{dx}f(x) - \frac{d}{dx}g(x)$
		\end{center}
		\subsubsection{Product Rule}
		If f and g are both differentiable: 
		\begin{center}
			$\frac{d}{dx}[f(x)g(x)] = f(x)\frac{d}{dx}[g(x)] + g(x)\frac{d}{dx}[f(x)]$
		\end{center}
		\subsubsection{Quotient Rule}
		If f and g are differentiable:
		\begin{center}
			$\frac{d}{dx}[\frac{f(x)}{g(x)}] = \frac{g(x)\frac{d}{dx}[f(x)]-f(x)\frac{d}{dx}[g(x)]}{[g(x)]^{2}}$
		\end{center}
		\subsection{The Product and Quotient Rules}
		\subsection{Derivatives of Trigonometric Functions}
			$\frac{d}{dx}(\sin{x}) = \cos{x}$
		\\	$\frac{d}{dx}(\cos{x}) = -\sin{x}$
		\\	$\frac{d}{dx}(\tan{x}) = \sec^{2}{x}$
		\\	$\frac{d}{dx}(\csc{x}) = \csc{x}\cot{x}$
		\\	$\frac{d}{dx}(\sec{x})= \sec{x}\tan{x}$
		\\	$\frac{d}{dx}(\cot{x}) = -\csc^{2}{x}$
		\\	$\frac{d}{dx}(\sin^{-1}{x}) = \frac{1}{\sqrt{1-x^{2}}}$
		\\	$\frac{d}{dx}(\cos^{-1}{x}) = -\frac{1}{\sqrt{1-x^{2}}}$
		\\	$\frac{d}{dx}(\tan^{-1}{x}) = \frac{1}{1+x^{2}}$
		\\ $\frac{d}{dx}(\cot^{-1}{x}) = -\frac{1}{1+x^{2}}$
		\\ $\frac{d}{dx}(\sec^{-1}{x}) = \frac{1}{x\sqrt{x^{2}-1}}$
		\\ $\frac{d}{dx}(\csc^{-1}{x}) = -\frac{1}{x\sqrt{x^{2}-1}}$

	
	\subsection{Chain Rule}
		Given the function $\frac{d}{dx}f(g(x))$, $ \frac{d}{dx} = f'(g(x))*g'(x)$

	\subsection{Chain Rule}
		If \textit{g} is differentiable at \textit{x} and \textit{f} is differentiable at $g(x)$ then the composite function $F = f*g$ defined by $F(x) = f(g(x))$ is differentiable at $x$ and that $F'$is given by the product:
		\begin{center}
			$F'(x) = f'(g(X)) * g'(x)$	
		\end{center}
		In Leibniz notation, if $y=f(u)$ and $u=g(x)$ are both differentiable functions then:
		\begin{center}
			$\frac{dy}{dx} = \frac{dy}{du}\frac{du}{dx}$
		\end{center}
		\begin{theorem}
			\textbf{Power Rule Combined With Chain Rule} If \textit{n} is any real number and $u=g(x)$ is differentiable then,
			\begin{center}
				$\frac{d}{dx}(u^{n}) = nu^{n-1}\frac{du}{dx}$
			\end{center}
			As such, $\frac{d}{dx}[g(x)]^{n} = n[g(x)]^{n-1} * g'(x)$
		\end{theorem}
	\subsection{Implicit Differentiation}
	\begin{definition}
		\textbf{Implicit Differentiation}:
		\\ The process of differentiating both sides of the equation with respect to x and then solving the resulting equation for $y'$
	\end{definition}
	Essentially, if the function says to take the derivative with respect to x, and there is a variable other than x present in the equation, derive the equation using normal differentiation rules, and at a $\frac{dn}{dx}$ where n is the variable letter present in the equation
	\subsection{Derivatives of Logarithmic Functions}
		General Form: $\frac{d}{dx}(log_{b}{x}) = \frac{1}{x\ln{b}}$
		Proof: Let $y=\log_{b}{x}$ Then
		\begin{center}
			$b^{y}=x$
		\end{center}
		Using the general form for taking the derivative, you get:
		\begin{center}
			$b^{y}(\ln{b})\frac{dy}{dx}=1$
			\\so
			\\$\frac{dy}{dx}=\frac{1}{b^{y}\ln{b}} = \frac{1}{x\ln{b}}$
		\end{center}
		Taking the Derivative of $y=\ln{x}$:
		\\Using the above General Form, you can solve for the derivative of $y=\ln{x}$:
		\begin{center}
			$\frac{d}{dx}(\ln{x})=\frac{1}{x}$
		\end{center}
		\textbf{IMPORTANT}:
		\\When taking the derivative of a logarithmic function, such as the derivative of $y=\ln{x^{3}+1}$ you will have to use chain rule.
		\\$\frac{d}{dx}(\ln{u}) = \frac{1}{u}\frac{du}{dx}$ and $\frac{d}{dx}[\ln{g(x)}] = \frac{g'(x)}{g(x)}$
		\\Another Derivative to remember is the derivative of $\ln|x|$ which derives to $\frac{1}{x}$
		\subsubsection{Using Logarithmic Differentiation}
		1. Take natural logarithms of both sides of an equation $y=f(x)$ and use the Laws of Logarithms to simplify
		\\2. Differentiate implicitly with respect to x.
		\\3. Splve the resulting equation for $y'$
		For example, let $y=x^{n}$
		\begin{center}
			$\ln{|y|} = \ln{|x|^{n}}=n\ln{x}$ when $x \neq 0$
		\end{center}
		For Logarithmic differentiation to be used, both the \textbf{base} and the \textbf{exponent} must not be constants
		\begin{definition}
			$e=\lim_{x\rightarrow 0}{(1+x)^{\frac{1}{x}}}$ 
			\\which when evaluated, defines e as $e\approx 2.7182818$
			\\Another form to define e is: 
			$e=\lim_{n\rightarrow\infty}{(1+\frac{1}{n})^{n}}$
		\end{definition}
	\subsection{Rates of Change in the Natural and Social Sciences}
	Rates of change: 
	\\If x changes from $\Delta{x} = x_{2}-x_{1}$
	\\Which means the change in y is $\Delta{y} = f(x_{2})-f(x_{1})$
	\\The difference quotient, which will give us the average rate of change in y with respect to x is defined as
	\begin{center} 
		$\frac{\Delta{y}}{\Delta{x}}= \frac{f(x_{2})-f(x_{1})}{x_{2}-x_{1}}$
	\end{center}
	Finding the instantaneous rate of change of y with respect to x is defined as:
	\begin{center}
		$\frac{dy}{dx}= \lim_{\Delta{x}\rightarrow 0}{\frac{\Delta{y}}{\Delta{x}}}$
	\end{center}
\end{document}