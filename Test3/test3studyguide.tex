\documentclass[10pt,a4paper]{article}
\usepackage[utf8]{inputenc}
\usepackage[T1]{fontenc}
\usepackage{amssymb}
\usepackage{graphicx}
\usepackage{mathtools}
\usepackage{amsthm}
\newtheorem{theorem}{Theorem}
\newtheorem{definition}{Definition}

\begin{document}
	\section{Chapter 4}
	\subsection{Minimum and Maximum Values}
	\begin{definition}
		Let $c$ be a number in the domain $D$ of a function $f$. Then $f(c)$ is the:
		\begin{center}
			\textbf{Absolute maximum} value of $f$ on $D$ if $f(c) \geq f(x) $ for all $x$ in $D$.
			\\		\textbf{Absolute minimum} value of $f$ on $D$ if $f(c) \leq f(x) $ for all $x$ in $D$.
		\end{center}
	\end{definition}
	\begin{definition}
		The number $f(c)$ is a
		\begin{center}
			\textbf{Local Maximum} value of $f$ if $f(c) \geq f(x)$ when $x$ is near $c$.
			\\		\textbf{Local Minimum} values of $f$ if $f(c) \leq f(x)$ when $x$ is near $c$.
		\end{center}
	\end{definition}
	\begin{theorem}
		\textbf{Extreme Value Theorem}
		\\	If $f$ is continuous on a closed interval $[a,b]$ then $f$ attains an absolute maximum values $f(c)$ and an absolute minimum value $f(d)$ at some numbers $c$ and $d$ in $[a,b]$.
	\end{theorem}
	\begin{theorem}
		\textbf{Fermat's Theorem}
		\\	If $f$ has a local maximum or minimum at $c$, and if $f'(c)$ exists, then $f'(c)=o$.
		\\If $f$ has a local maximum or minimum at $c$, then $c$ is a critical number of $f$.
	\end{theorem}
	\begin{definition}
		A \textbf{critical number} of a function $f$ is a number $c$ c in the domain of $f$ such that either $f'(c)=0$ or $f'(c)$ does not exist.
	\end{definition}
	\subsubsection{Finding the Absolute Max and Min Values using the Closed Interval Method}
	To find the \textit{absolute} maximum and minimum values of a continuous function $f$ on a closed interval $[a,b]$:
	\\1. Find the values of $f$ at the critical numbers of $f$ in $(a,b)$.
	\\2. Find the values of $f$ at the endpoints of the interval.
	\\3. The largest of the values from Step 1 and 2 is the absolute maximum value; the smallest of these values is the absolute minimum value.
	
	\begin{theorem}
		\textbf{Role's Theorem}:
		\\Let \textit{f} be a function that satisfies the following three hypotheses:
		\\1. \textit{f} is continuous on the closed interval $[a,b]$
		\\2. \textit{f} is differentiable on the open interval $(a,b)$
		\\3. $f(a) = f(b)$
		\\ Then there is a number \textit{c} in $(a,b)$ such that $f'(c)=0$	
	\end{theorem}
	\begin{theorem}
		\textbf{Mean Value Theorem}:
		\\Let $f$ be a function that satisfies the following hypotheses:
		\\1. $f$ is continuous on the closed interval $[a,b]$
		\\2. $f$ is differentiable on the open interval $(a,b)$
		\\ Then there is a number $c$ in $(a,b)$ such that:
		\begin{center}
			$f'(c)= \frac{f(b)-f(a)}{b-a}$
			\\or equivalently
			\\$f(b)-f(a) = f'(c)(b-a)$
		\end{center}
	\end{theorem}
	The slope of the secant line $AB$ is $m_{ab}=\frac{f(b)-f(a)}{b-a}$
	$\frac{d}{dx}(\sin{x}) = \cos{x}$
	\\	$\frac{d}{dx}(\cos{x}) = -\sin{x}$
	\\	$\frac{d}{dx}(\tan{x}) = \sec^{2}{x}$
	\\	$\frac{d}{dx}(\csc{x}) = \csc{x}\cot{x}$
	\\	$\frac{d}{dx}(\sec{x})= \sec{x}\tan{x}$
	\\	$\frac{d}{dx}(\cot{x}) = -\csc^{2}{x}$
	\\	$\frac{d}{dx}(\sin^{-1}{x}) = \frac{1}{\sqrt{1-x^{2}}}$
	\\	$\frac{d}{dx}(\cos^{-1}{x}) = -\frac{1}{\sqrt{1-x^{2}}}$
	\\	$\frac{d}{dx}(\tan^{-1}{x}) = \frac{1}{1+x^{2}}$
	\\ $\frac{d}{dx}(\cot^{-1}{x}) = -\frac{1}{1+x^{2}}$
	\\ $\frac{d}{dx}(\sec^{-1}{x}) = \frac{1}{x\sqrt{x^{2}-1}}$
	\\ $\frac{d}{dx}(\csc^{-1}{x}) = -\frac{1}{x\sqrt{x^{2}-1}}$
	General Form: $\frac{d}{dx}(log_{b}{x}) = \frac{1}{x\ln{b}}$
	Proof: Let $y=\log_{b}{x}$ Then
	\begin{center}
		$b^{y}=x$
	\end{center}
	Using the general form for taking the derivative, you get:
	\begin{center}
		$b^{y}(\ln{b})\frac{dy}{dx}=1$
		\\so
		\\$\frac{dy}{dx}=\frac{1}{b^{y}\ln{b}} = \frac{1}{x\ln{b}}$
	\end{center}
	Taking the Derivative of $y=\ln{x}$:
	\\Using the above General Form, you can solve for the derivative of $y=\ln{x}$:
	\begin{center}
		$\frac{d}{dx}(\ln{x})=\frac{1}{x}$
	\end{center}
	\textbf{IMPORTANT}:
	\\When taking the derivative of a logarithmic function, such as the derivative of $y=\ln{x^{3}+1}$ you will have to use chain rule.
	\\$\frac{d}{dx}(\ln{u}) = \frac{1}{u}\frac{du}{dx}$ and $\frac{d}{dx}[\ln{g(x)}] = \frac{g'(x)}{g(x)}$
	\\Another Derivative to remember is the derivative of $\ln|x|$ which derives to $\frac{1}{x}$
	1. Take natural logarithms of both sides of an equation $y=f(x)$ and use the Laws of Logarithms to simplify
	\\2. Differentiate implicitly with respect to x.
	\\3. Splve the resulting equation for $y'$
	For example, let $y=x^{n}$
	\begin{center}
		$\ln{|y|} = \ln{|x|^{n}}=n\ln{x}$ when $x \neq 0$
	\end{center}
	For Logarithmic differentiation to be used, both the \textbf{base} and the \textbf{exponent} must not be constants
	\begin{definition}
		$e=\lim_{x\rightarrow 0}{(1+x)^{\frac{1}{x}}}$ 
		\\which when evaluated, defines e as $e\approx 2.7182818$
		\\Another form to define e is: 
		$e=\lim_{n\rightarrow\infty}{(1+\frac{1}{n})^{n}}$
	\end{definition}
\end{document}